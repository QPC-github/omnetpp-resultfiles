\section{How to transform the run attributes into 'wide' format?}

The omnetpp dataset contains the run attributes in 'long' format:

\begin{verbatim}
> d$runattrs
                                          runid       attrname                            attrvalue
1 PureAlohaExperiment-0-20090220-15:51:12-31531     configname                  PureAlohaExperiment
2 PureAlohaExperiment-0-20090220-15:51:12-31531       datetime                    20090220-15:51:12
3 PureAlohaExperiment-0-20090220-15:51:12-31531     experiment                  PureAlohaExperiment
4 PureAlohaExperiment-0-20090220-15:51:12-31531        inifile                          omnetpp.ini
5 PureAlohaExperiment-0-20090220-15:51:12-31531  iterationvars                $numHosts=10, $mean=1
6 PureAlohaExperiment-0-20090220-15:51:12-31531 iterationvars2 $numHosts=10, $mean=1, $repetition=0
...
\end{verbatim}

If you want to transform this into a 'wide' format (one column per attribute),
you can use the \ttt{cast} function from the \ttt{reshape} package:

\begin{verbatim}
> cast(d$runattrs, runid~attrname, value='attrvalue')
                                           runid          configname          datetime          experiment     inifile         iterationvars                       iterationvars2 mean
1  PureAlohaExperiment-0-20090220-15:51:12-31531 PureAlohaExperiment 20090220-15:51:12 PureAlohaExperiment omnetpp.ini $numHosts=10, $mean=1 $numHosts=10, $mean=1, $repetition=0    1
2 PureAlohaExperiment-10-20090220-15:51:14-31542 PureAlohaExperiment 20090220-15:51:14 PureAlohaExperiment omnetpp.ini $numHosts=10, $mean=7 $numHosts=10, $mean=7, $repetition=0    7
3 PureAlohaExperiment-11-20090220-15:51:14-31543 PureAlohaExperiment 20090220-15:51:14 PureAlohaExperiment omnetpp.ini $numHosts=10, $mean=7 $numHosts=10, $mean=7, $repetition=1    7
4  PureAlohaExperiment-1-20090220-15:51:12-31532 PureAlohaExperiment 20090220-15:51:12 PureAlohaExperiment omnetpp.ini $numHosts=10, $mean=1 $numHosts=10, $mean=1, $repetition=1    1
5 PureAlohaExperiment-12-20090220-15:51:14-31544 PureAlohaExperiment 20090220-15:51:14 PureAlohaExperiment omnetpp.ini $numHosts=10, $mean=9 $numHosts=10, $mean=9, $repetition=0    9
6 PureAlohaExperiment-13-20090220-15:51:14-31545 PureAlohaExperiment 20090220-15:51:14 PureAlohaExperiment omnetpp.ini $numHosts=10, $mean=9 $numHosts=10, $mean=9, $repetition=1    9
            measurement network numHosts processid repetition replication resultdir runnumber seedset
1 $numHosts=10, $mean=1   Aloha       10     31531          0          #0   results         0       0
2 $numHosts=10, $mean=7   Aloha       10     31542          0          #0   results        10      10
3 $numHosts=10, $mean=7   Aloha       10     31543          1          #1   results        11      11
4 $numHosts=10, $mean=1   Aloha       10     31532          1          #1   results         1       1
5 $numHosts=10, $mean=9   Aloha       10     31544          0          #0   results        12      12
6 $numHosts=10, $mean=9   Aloha       10     31545          1          #1   results        13      13
...
\end{verbatim}

If you want to restrict the attribute columns:

\begin{verbatim}
> cast(d$runattrs, runid~attrname, value='attrvalue', subset=attrname %in% c('experiment','measurement','replication'))
                                           runid          experiment           measurement replication
1  PureAlohaExperiment-0-20090220-15:51:12-31531 PureAlohaExperiment $numHosts=10, $mean=1          #0
2 PureAlohaExperiment-10-20090220-15:51:14-31542 PureAlohaExperiment $numHosts=10, $mean=7          #0
3 PureAlohaExperiment-11-20090220-15:51:14-31543 PureAlohaExperiment $numHosts=10, $mean=7          #1
4  PureAlohaExperiment-1-20090220-15:51:12-31532 PureAlohaExperiment $numHosts=10, $mean=1          #1
5 PureAlohaExperiment-12-20090220-15:51:14-31544 PureAlohaExperiment $numHosts=10, $mean=9          #0
6 PureAlohaExperiment-13-20090220-15:51:14-31545 PureAlohaExperiment $numHosts=10, $mean=9          #1
...
\end{verbatim}

\section{How to transform a dataset into another dataset where all scalars are averaged across replications?}

First add the \ttt{experiment} and \ttt{measurement} columns to the scalars:

\begin{verbatim}
> scalars <- merge(d$scalars,
                   cast(d$runattrs, runid~attrname, value='attrvalue', subset=attrname %in% c('experiment','measurement')),
                   by='runid',
                   all.x=TRUE)
\end{verbatim}

Now you can compute the averages by calling \ttt{cast()} again:

\begin{verbatim}
> cast(scalars, experiment+measurement+module+name ~ ., mean)
\end{verbatim}

To compute confidence intervals as well:

\begin{verbatim}
> ci95 <- conf.int(0.95)
> cast(scalars, experiment+measurement+module+name ~ ., c(mean,ci95))
\end{verbatim}

\section{How to average some scalars across modules?}

For example, if you want the average of \ttt{'avgQueueLength'} across all queue modules:

\begin{verbatim}
> ql <- subset(d$scalars, grepl('.*\\.queue', module) & name='avgQueueLength')
> mean(ql$value)
\end{verbatim}

\section{How to sum scalars grouped by modules?}

Assume you want to compute the sum of drop counts for each node in the
omnetpp/samples/routing/Net60 network.

First load the scalars generated by the simulation:

\begin{verbatim}
> d <- loadDataset('/home/tomi/work/omnetpp/samples/routing/results/Net60-0.sca')
\end{verbatim}

Each node may have several input gates and a message queue for each input port.
You can select the drop counts of the queues by:

\begin{verbatim}
> dc <- subset(d$scalars, name=='drop:count' & grepl('.*queue.*',module))
> dc

     resultkey                          runid                                                        file                 module       name value
10           9 Net60-0-20100816-10:48:41-7648 /home/tomi/work/omnetpp/samples/routing/results/Net60-0.sca  Net60.rte[0].queue[0] drop:count     0
21          20 Net60-0-20100816-10:48:41-7648 /home/tomi/work/omnetpp/samples/routing/results/Net60-0.sca  Net60.rte[1].queue[0] drop:count     0
30          29 Net60-0-20100816-10:48:41-7648 /home/tomi/work/omnetpp/samples/routing/results/Net60-0.sca  Net60.rte[1].queue[1] drop:count     0
39          38 Net60-0-20100816-10:48:41-7648 /home/tomi/work/omnetpp/samples/routing/results/Net60-0.sca  Net60.rte[1].queue[2] drop:count     0
50          49 Net60-0-20100816-10:48:41-7648 /home/tomi/work/omnetpp/samples/routing/results/Net60-0.sca  Net60.rte[2].queue[0] drop:count     0
61          60 Net60-0-20100816-10:48:41-7648 /home/tomi/work/omnetpp/samples/routing/results/Net60-0.sca  Net60.rte[3].queue[0] drop:count     0
72          71 Net60-0-20100816-10:48:41-7648 /home/tomi/work/omnetpp/samples/routing/results/Net60-0.sca  Net60.rte[4].queue[0] drop:count    12
81          80 Net60-0-20100816-10:48:41-7648 /home/tomi/work/omnetpp/samples/routing/results/Net60-0.sca  Net60.rte[4].queue[1] drop:count     0
90          89 Net60-0-20100816-10:48:41-7648 /home/tomi/work/omnetpp/samples/routing/results/Net60-0.sca  Net60.rte[4].queue[2] drop:count     0
99          98 Net60-0-20100816-10:48:41-7648 /home/tomi/work/omnetpp/samples/routing/results/Net60-0.sca  Net60.rte[4].queue[3] drop:count     0
...
\end{verbatim}

To sum the drop counts for each node:

\begin{verbatim}
> r <- aggregate(list(value=dc$value),
                 list(runid=dc$runid,file=dc$file,module=sub('\\.queue\\[[0-9]+\\]', '', dc$module),name=dc$name),
                 sum)
> head(r)
                           runid                                                        file        module       name value
1 Net60-0-20100816-10:48:41-7648 /home/tomi/work/omnetpp/samples/routing/results/Net60-0.sca  Net60.rte[0] drop:count     0
2 Net60-0-20100816-10:48:41-7648 /home/tomi/work/omnetpp/samples/routing/results/Net60-0.sca  Net60.rte[1] drop:count     0
3 Net60-0-20100816-10:48:41-7648 /home/tomi/work/omnetpp/samples/routing/results/Net60-0.sca Net60.rte[10] drop:count     6
4 Net60-0-20100816-10:48:41-7648 /home/tomi/work/omnetpp/samples/routing/results/Net60-0.sca Net60.rte[11] drop:count     0
5 Net60-0-20100816-10:48:41-7648 /home/tomi/work/omnetpp/samples/routing/results/Net60-0.sca Net60.rte[12] drop:count     0
6 Net60-0-20100816-10:48:41-7648 /home/tomi/work/omnetpp/samples/routing/results/Net60-0.sca Net60.rte[13] drop:count     0

\end{verbatim}

If you want to add these scalars to the original dataset, first generate \ttt{resultkey} for the new scalars,
and use \ttt{rbind}:

\begin{verbatim}
> r <- cbind(list(resultkey=seq(from=max(d$scalars$resultkey, d$vectors$resultkey, d$histograms$resultkey) + 1,
                                length.out=nrow(r))),
             r)
> d$scalars <- rbind(d$scalars, r)
\end{verbatim}

\section{How to compute the mean, stddev, min, max etc of vectors?}

The current version of the \ttt{omnetpp} package does not support computing
the summary statistics without loading the vectors, but it is planned in a future
version to do so.

So first load the vectors into the memory:

\begin{verbatim}
> d <- loadDataset('Aloha.vec', add('vector'))
> d <- loadVectors(d, NULL)
\end{verbatim}

\subsection{Using the \ttt{base} package}

If the vector is loaded into a dataset, then the \ttt{mean}, \ttt{sd}, \ttt{min}, \ttt{max}
functions can be used:

\begin{verbatim}
> vs <- split(d$vectordata$y, d$vectordata$resultkey)
> lapply(vs, mean)
> lapply(vs, sd)
> lapply(vs, min)
> lapply(vs, max)
> lapply(vs, summary)
\end{verbatim}

\subsection{Using the \ttt{reshape} package}

Its much simpler:

\begin{verbatim}
> cast(d$vectordata, resultkey~., c(mean,sd,min,max), value='y')
\end{verbatim}

\subsection{Using the \ttt{doBy} package}

If you have the \ttt{doBy} package installed:

\begin{verbatim}
> require(doBy)
> summaryBy(y ~ resultkey, d$vectordata, FUN = list(mean, sd, min, max))
\end{verbatim}


\section{How to plot a vector?}

\subsection{Using the \ttt{omnetpp} package}

If \ttt{v} a matrix or data frame with \ttt{x} and \ttt{y} columns, you can draw a plot as:

\begin{verbatim}
> plotLineChart(list(v))
\end{verbatim}

\subsection{Using the \ttt{lattice} package}

Another possibility is to use the \ttt{xyplot} function from the \ttt{lattice} package:

\begin{verbatim}
> xyplot(y~x, d$vectordata, subset=resultkey==2, type='l')
\end{verbatim}

\section{How to plot several vectors?}

If \ttt{v1}, \\ttt{v2}, \ttt{v3} are matrices or data frames with \ttt{x} and \ttt{y} columns,
the plot containing the 3 lines can be generated by:

\begin{verbatim}
> plotLineChart(list(a=v1, b=v2, c=v3))
\end{verbatim}

The lines are named \ttt{'a'}, \ttt{'b'} and \ttt{'c'}, these names can be used to specify
properties of the lines:

\begin{verbatim}
> plotLineChart(list(a=v1, b=v2, c=v3), 'Line.Color/a'='red', 'Symbols.Type/b'='Square', 'Line.Type/c'='SampleHold')
\end{verbatim}

These line names are displayed on the legend too:

\begin{verbatim}
> plotLineChart(list(a=v1, b=v2, c=v3), Legend.Display='true')
\end{verbatim}

If you want to generate a plot from vectors in a dataset, use the \ttt{makeLineChartDataset()} to select
the vectors and give the lines sensible names:

\begin{verbatim}
> vs <- makeLineChartDataset(dataset, '\${configname}/\${runnumber} - \${module} \${name}')
> plotLineChart(vs, Legend.Display='true')
\end{verbatim}

\subsection{Using the \ttt{lattice} package}

\begin{verbatim}
> xyplot(y~x, d$vectordata, groups=resultkey)
\end{verbatim}

\section{How to compute and plot a histogram of a vector?}

\subsection{Using the \ttt{graphics} package}

If \ttt{v} is a numeric vector, then the \ttt{hist()} function from the \ttt{graphics} package
can be used to generate a histogram object and draw a histograme plot.

\begin{verbatim}
> h <- hist(v)
\end{verbatim}

If you want to put several histogram on a single plot:

\begin{verbatim}
> h1 <- hist(v1, plot=FALSE)
> h2 <- hist(v2, plot=FALSE)
> plot(h1)
> plot(h2, add=TRUE)
\end{verbatim}

\subsection{Using the \ttt{lattice} package}

\begin{verbatim}
> histogram(~y, d$vectordata, groups=resultkey)
\end{verbatim}

\section{How to plot the histograms loaded from scalar files?}

If \ttt{d} is a loaded dataset containing histogram data, then \ttt{makeHistograms()} can generate
histogram objects:

\begin{verbatim}
> hs <- makeHistograms(d, '{module} {name}') 
\end{verbatim}

The result is a list of histgram objects, the elements are identified by module and statistic name.

\section{How to plot a histogram together with its cumulative distribution function?}

TODO

\section{How to create a bar chart of some scalars where error bars indicate confidence interval?}

\begin{verbatim}
> heigths <- makeBarChartDataset(dataset, rows=c('measurement'), columns=c('name'))
> conf.intervals <- makeBarChartDataset(dataset, rows=c('measurement'), columns=c('name'), conf.int(0.95) )
> plotBarChart(heigths, conf.int=conf.intervals, Legend.Display='true', Legend.Anchoring='NorthWest')
\end{verbatim}

\section{How to plot a scatter chart where the X axis is some run iteration
variable (i.e. \$numHosts)?}

Iteration variables are saved as scalars as well, so you can write:

\begin{verbatim}
> d <- makeScatterChartDataset(dataset, xModule='.', xName='numHosts')
> plotLineChart(d)
\end{verbatim}

\section{How to plot a 3D chart where the X,Y axes are run iteration variables?}

In this example we create a 3D plot of the \ttt{total receive time} as a function
of the mean \ttt{iaTime} and \ttt{numOfHosts} from the Aloha simulation.

First create a data frame containing the values of the scalars for each run:
 
\begin{verbatim}
> d <- loadDataset('/home/tomi/work/omnetpp-resultfiles/R-package/inst/extdata/PureAlohaExperiment-*.sca')
> data3d <- cast(d$scalars, runid ~ name, subset=name%in%c('mean','numHosts','total receive time'))
\end{verbatim}

Now you can use the \ttt{lattice} package to draw a 3D scatter plot (\ttt{cloud()}) or a
3D surface plot (\ttt{wireframe()}):

\begin{verbatim}
> require(lattice)
> cloud(`total receive time` ~ numHosts * mean, data=data3d)
\end{verbatim}

\image{http://github.com/omnetpp/omnetpp-resultfiles/raw/tomi/R-package/tutorial/figures/tips/cloud.png}


