\title{Omnetpp R Package Tutorial}
\center{author: Andras Varga}


\section{Introduction}

This is a tutorial for the "omnetpp" R package. R (http://r-project.org) is a 
powerful open-source software environment for statistical computing and graphics,
and the "omnetpp" package was written to facilitate evaluating OMNeT++ simulation 
results in R. The package supports loading the contents of OMNeT++ result files 
into R, organizing the data and creating various plots from them.

This tutorial was meant primarily for R newbies, as R can be intimidating at first.
Experienced R users may find that I am explaining some basic R syntax and concepts 
they are already familiar with.


\section{Getting Started}

First, install the package into R (note the capital letters!):

\begin{verbatim}
$ R CMD INSTALL omnetpp_0.1-1.tar.gz
\end{verbatim}

If you ever need to deinstall it, you can do that with

\begin{verbatim}
$ R CMD REMOVE omnetpp
\end{verbatim}

Start R:

\begin{verbatim}
$ R
\end{verbatim}

and load the OMNeT++ extension library by typing the following command at the R prompt (">"):

\begin{verbatim}
> require(omnetpp)
Loading required package: omnetpp
\end{verbatim}

Getting help:

\begin{verbatim}
> ?omnetpp
\end{verbatim}

displays a short description of the package, and hints that the

\begin{verbatim}
> library(help="omnetpp")
\end{verbatim}

command lists all commands contributed by this package:

\begin{verbatim}
> library(help="omnetpp")

                Information on package 'omnetpp'

Description:

Package:       omnetpp
Version:       0.1-1
<snip>

Index:

filters                 Filters
loadDataset             Loads data from result files
loadVectors             Loads vector data from result files and applies
                        some processing
makeBarChartDataset     Create a bar chart dataset from scalars
makeHistograms          Build histogram objects from histogram bins.
makeScatterChartDataset Create XY data from scalars.
omnetpp-package         The OMNeT++ Package
patterns                Pattern language
plotBarChart            Plots a bar chart.
plotHistogramChart      Plot a histogram chart.
plotLineChart           Plots a line chart.
\end{verbatim}

You can also watch the demos:

\begin{verbatim}
> demo(dataset, package='omnetpp')
> demo(charts, package='omnetpp')
\end{verbatim}

For this tutorial, I gathered some OMNeT++ result files in the directory
in the \ttt{rtut/} subdir of my home directory. Let us change first into that
directory so that we can access the files without having to specify
full path. This is possible by quitting R (\ttt{q()} command or Ctrl+D),
changing into that directory at the shell prompt and restarting R,
but it can also be done from within R:

\begin{verbatim}
> setwd("~/rtut")
> getwd()
[1] "/home/andras/rtut"
> list.files()
[1] "OneFifo-0.vci"             "OneFifo-0.vec"
[3] "PureAlohaExperiment-0.sca" "PureAlohaExperiment-1.sca"
[5] "PureAlohaExperiment-2.sca" "TokenRing1-0.vci"
[7] "TokenRing1-0.vec"
\end{verbatim}

Note that the dot is just a normal character in R without a special meaning:
\ttt{list.files()} is just a plain identifier, and the function could as well
have been called \ttt{list_files()} or \ttt{listFiles()}. The "member selector"
operator (which is the dot character in C and many other languages) is the
dollar character (\ttt{\$}) in R.

Now, we load the contents of some result files into an R data structure, using
the \ttt{loadDataset()} command from the omnetpp package:

\begin{verbatim}
> x <- loadDataset("*.sca")
\end{verbatim}

\ttt{loadDataset()}'s first argument can also be a string vector, which allows you
to specify several file names or wildcard patterns:

\begin{verbatim}
> tmp <- loadDataset( c("*.sca", "OneFifo-*.vec", "TokenRing1-*.vec") )
\end{verbatim}

where \ttt{c()} is the R syntax for combining several values into a vector.
\ttt{loadDataset()} also lets you specify filter criteria if you only want to load
a subset of the data from the files. You can filter by module name, result item name
and other criteria; will have a look at this feature later.


Let's have a look at the x variable that contains the dataset we loaded from \ttt{"*.sca"}.
We can print its contents by simply typing the variable's name:

\begin{verbatim}
> x
...
\end{verbatim}

The output is quite long. However, you can notice that it is a list of several
components, where each component is a table that R calls "data frame".
The tables are named "scalars", "vectors", "fields", "fileruns", etc.
The names can be displayed with the

\begin{verbatim}
> names(x)
[1] "runattrs"   "fileruns"   "scalars"    "vectors"    "statistics"
[6] "fields"     "bins"       "params"     "attrs"
\end{verbatim}

command, and individual components can be printed with the \$ notation:

\begin{verbatim}
> x$fileruns
                                          runid                      file
1 PureAlohaExperiment-0-20100427-16:30:15-25274 PureAlohaExperiment-0.sca
2 PureAlohaExperiment-2-20100427-16:30:17-25289 PureAlohaExperiment-2.sca
3 PureAlohaExperiment-1-20100427-16:30:16-25283 PureAlohaExperiment-1.sca

> x$scalars
   resultkey                                         runid                      file       module                    name        value
1          0 PureAlohaExperiment-0-20100427-16:30:15-25274 PureAlohaExperiment-0.sca            .                    mean 1.000000e+00
2          1 PureAlohaExperiment-0-20100427-16:30:15-25274 PureAlohaExperiment-0.sca            .                numHosts 1.000000e+01
3          2 PureAlohaExperiment-0-20100427-16:30:15-25274 PureAlohaExperiment-0.sca Aloha.server                duration 5.400007e+03
4          3 PureAlohaExperiment-0-20100427-16:30:15-25274 PureAlohaExperiment-0.sca Aloha.server    collisionLength:mean 1.969656e-01
5          4 PureAlohaExperiment-0-20100427-16:30:15-25274 PureAlohaExperiment-0.sca Aloha.server     collisionLength:sum 2.415980e+03
6          5 PureAlohaExperiment-0-20100427-16:30:15-25274 PureAlohaExperiment-0.sca Aloha.server     collisionLength:max 8.734217e-01
7          6 PureAlohaExperiment-0-20100427-16:30:15-25274 PureAlohaExperiment-0.sca Aloha.server     collidedFrames:last 3.992800e+04
...

> x$attrs
    attrtype resultkey          attrname                         attrvalue
1     scalar         3             title            collision length, mean
2     scalar         4             title             collision length, sum
3     scalar         5             title             collision length, max
4     scalar         6            source                    sum(collision)
5     scalar         6             title             collided frames, last
6     scalar         7 interpolationmode                            linear
7     scalar         7            source                  timeavg(receive)
...
\end{verbatim}

You may notice that this data structure looks like a relational database, with
the "runid" and "resultkey" colums being keys that relate tables to each other.
Indeed, R makes it simple to "join" tables (in the SQL sense); we'll show this later.

If you are worried about the memory consumption of the above tables due to many
repeated copies of run IDs, file names, module names, etc, you generally shouldn't be:
R internally stores the distinct values in a lookup table, and the data frame's rows
only contain integers that are indices into the internal tables.

At this point, R's default display width of 80 characters may have resulted in wrapping
of the tables. Some R versions automatically adapt the display width to the width of the
terminal window, but if yours doesn't, you can increase the display width with the
following command:

\begin{verbatim}
> options(width=120)
\end{verbatim}



\section{Exploring the Loaded Scalars}

In the section above, we have loaded a bunch of scalar files with the

\begin{verbatim}
> x <- loadDataset("*.sca")
\end{verbatim}

command into the variable x, and seen that scalar values are in its "scalars" data frame,
\ttt{x\$scalars}. Let's examine the scalar values a little!

Individual columns of a data frame can also be selected with the \$ notation,
so the values of all output scalars can be displayed with \ttt{x\$scalars\$value}:

\begin{verbatim}
> names(x$scalars)
[1] "resultkey" "runid"     "file"      "module"    "name"      "value"

> x$scalars$value
 [1] 1.000000e+00 1.000000e+01 5.400007e+03 1.969656e-01 2.415980e+03
 [6] 8.734217e-01 3.992800e+04 1.612558e-01 8.781000e+03 1.000000e+00
[11] 1.000000e+01 5.400023e+03 1.980867e-01 2.449936e+03 1.193256e+00
[16] 4.054700e+04 1.582669e-01 8.618000e+03 2.000000e+00 1.000000e+01
[21] 5.400176e+03 1.682003e-01 9.881766e+02 6.155766e-01 1.463500e+04
[26] 1.944002e-01 1.058300e+04
\end{verbatim}

We store the values in a separate variable:

\begin{verbatim}
> v <- x$scalars$value
\end{verbatim}

and calculate some statistical properties:

\begin{verbatim}
> min(v)
[1] 0.1582669
> max(v)
[1] 40547
> sum(v)
[1] 145184.1
> length(v)
[1] 27
> mean(v)
[1] 5377.187
> var(v)
[1] 116749219
> sd(v)
[1] 10805.06
\end{verbatim}

R also has a function that computes a statistical summary in one step:

\begin{verbatim}
> summary(v)
     Min.   1st Qu.    Median      Mean   3rd Qu.      Max.
1.583e-01 7.445e-01 1.000e+01 5.377e+03 5.400e+03 4.055e+04
\end{verbatim}

\ttt{v} is a normal vector variable. The index operator has the usual bracket
syntax (\ttt{[]}) and works as expected, except that indexing is 1-based
(not 0-based as in C), and ranges are also accepted:

\begin{verbatim}
> v[3]
[1] 5400.007
> 1:3
[1] 1 2 3
> v[1:3]
[1]    1.000   10.000 5400.007
\end{verbatim}

The index operator also accepts a boolean vector that selects which elements
of the vector you want to obtain. For example, the following example selects
all elements that are larger than 100:

\begin{verbatim}
> v > 100
 [1] FALSE FALSE  TRUE FALSE  TRUE FALSE  TRUE FALSE  TRUE FALSE FALSE  TRUE
[13] FALSE  TRUE FALSE  TRUE FALSE  TRUE FALSE FALSE  TRUE FALSE  TRUE FALSE
[25]  TRUE FALSE  TRUE
> v[v>100]
 [1]  5400.0073  2415.9798 39928.0000  8781.0000  5400.0235  2449.9358
 [7] 40547.0000  8618.0000  5400.1756   988.1766 14635.0000 10583.0000
\end{verbatim}

This feature can be used for more complex queries as well, like selecting those
elements that are within 90\% of the maximum:

\begin{verbatim}
> v[v>=0.9*max(v)]
[1] 39928 40547
\end{verbatim}

Of course, the above queries are not very meaningful, because we lumped together
all scalar result items, mixing apples with oranges. In an effort to fix this,
let's see first what names occur in our scalar results:

\begin{verbatim}
> levels(x$scalars$name)
[1] "channelUtilization:last" "collidedFrames:last"
[3] "collisionLength:max"     "collisionLength:mean"
[5] "collisionLength:sum"     "duration"
[7] "mean"                    "numHosts"
[9] "receivedFrames:last"
\end{verbatim}

To select scalars with a specific name (e.g. \ttt{receivedFrames:last}), we can use the
indexing operator again (note the comma before the close bracket!):

\begin{verbatim}
> receivedFrames <- x$scalars[x$scalars$name == "receivedFrames:last",]
> receivedFrames
   resultkey                                         runid                      file       module                name value
9          8 PureAlohaExperiment-0-20100427-16:30:15-25274 PureAlohaExperiment-0.sca Aloha.server receivedFrames:last  8781
18        17 PureAlohaExperiment-1-20100427-16:30:16-25283 PureAlohaExperiment-1.sca Aloha.server receivedFrames:last  8618
27        26 PureAlohaExperiment-2-20100427-16:30:17-25289 PureAlohaExperiment-2.sca Aloha.server receivedFrames:last 10583

> mean(receivedFrames$value)
[1] 9327.333
\end{verbatim}

The selection worked because rows of a data frame can also be addressed with the
\ttt{[]} operator (which accepts indices and a boolean array as well):

\begin{verbatim}
> x$scalars[5,]
  resultkey                                         runid                      file       module                name   value
5         4 PureAlohaExperiment-0-20100427-16:30:15-25274 PureAlohaExperiment-0.sca Aloha.server collisionLength:sum 2415.98
> \end{verbatim}

You can filter by module name, file name or run ID in the same way.


\section{Load-Time Dataset Filtering}

The \ttt{loadDataset()} function can do filtering on the fly, i.e. you can specify 
which data in the file(s) you want to be present in the result dataset. This is done
by giving a second argument to \ttt{loadDataset()}, a list of \ttt{add} and \ttt{discard}
nodes. Within an \ttt{add} or \ttt{discard} one can specify the result type (scalar, vector, 
statistic), and a filter pattern that can match result item names, module names, etc.
An example:

\begin{verbatim}
> loadDataset(c('Aloha-1.sca', 'Aloha-2.sca', 'PureAloha1-*.vec'),
       add(type='scalar', select='module(Aloha.server)'),
       add('vector'),
       discard('vector', 'name("channel utilization")')
     )
\end{verbatim}

The pattern language is the same that is used in various places in the IDE's Analysis Editor 
(e.g. in the Advanced Filter on the Browse Data page) and also in scavetool. You can get help
about the pattern language by typing \ttt{?patterns} in R.


\section{Reorganizing the Data}

R provides many interesting ways to reorganize data in data frames, which can often be handy.

The \ttt{split()} function groups data by some criteria, and produces a list of data frames.
The criteria can be a column name as well. For example, the following command groups our
scalars by runs:

\begin{verbatim}
> split(x$scalars, x$scalars$runid)
$`PureAlohaExperiment-0-20100427-16:30:15-25274`
  resultkey                                         runid                      file       module        name        value
1         0 PureAlohaExperiment-0-20100427-16:30:15-25274 PureAlohaExperiment-0.sca            .        mean 1.000000e+00
2         1 PureAlohaExperiment-0-20100427-16:30:15-25274 PureAlohaExperiment-0.sca            .    numHosts 1.000000e+01
3         2 PureAlohaExperiment-0-20100427-16:30:15-25274 PureAlohaExperiment-0.sca Aloha.server    duration 5.400007e+03
...

$`PureAlohaExperiment-1-20100427-16:30:16-25283`
   resultkey                                         runid                      file       module       name        value
10         9 PureAlohaExperiment-1-20100427-16:30:16-25283 PureAlohaExperiment-1.sca            .       mean 1.000000e+00
11        10 PureAlohaExperiment-1-20100427-16:30:16-25283 PureAlohaExperiment-1.sca            .   numHosts 1.000000e+01
12        11 PureAlohaExperiment-1-20100427-16:30:16-25283 PureAlohaExperiment-1.sca Aloha.server   duration 5.400023e+03
...

$`PureAlohaExperiment-2-20100427-16:30:17-25289`
   resultkey                                         runid                      file       module       name        value
19        18 PureAlohaExperiment-2-20100427-16:30:17-25289 PureAlohaExperiment-2.sca            .       mean 2.000000e+00
20        19 PureAlohaExperiment-2-20100427-16:30:17-25289 PureAlohaExperiment-2.sca            .   numHosts 1.000000e+01
21        20 PureAlohaExperiment-2-20100427-16:30:17-25289 PureAlohaExperiment-2.sca Aloha.server   duration 5.400176e+03
...
\end{verbatim}

Or we can split by module name, to see which statistics each module wrote:

\begin{verbatim}
> split(x$scalars, x$scalars$module)
$.
   resultkey                                         runid                      file module     name value
1          0 PureAlohaExperiment-0-20100427-16:30:15-25274 PureAlohaExperiment-0.sca      .     mean     1
2          1 PureAlohaExperiment-0-20100427-16:30:15-25274 PureAlohaExperiment-0.sca      . numHosts    10
10         9 PureAlohaExperiment-1-20100427-16:30:16-25283 PureAlohaExperiment-1.sca      .     mean     1
11        10 PureAlohaExperiment-1-20100427-16:30:16-25283 PureAlohaExperiment-1.sca      . numHosts    10
19        18 PureAlohaExperiment-2-20100427-16:30:17-25289 PureAlohaExperiment-2.sca      .     mean     2
20        19 PureAlohaExperiment-2-20100427-16:30:17-25289 PureAlohaExperiment-2.sca      . numHosts    10

$Aloha.server
   resultkey                                         runid                      file       module                    name        value
3          2 PureAlohaExperiment-0-20100427-16:30:15-25274 PureAlohaExperiment-0.sca Aloha.server                duration 5.400007e+03
4          3 PureAlohaExperiment-0-20100427-16:30:15-25274 PureAlohaExperiment-0.sca Aloha.server    collisionLength:mean 1.969656e-01
5          4 PureAlohaExperiment-0-20100427-16:30:15-25274 PureAlohaExperiment-0.sca Aloha.server     collisionLength:sum 2.415980e+03
...
\end{verbatim}

R also provides an \ttt{unsplit()} function that reverses the effect of \ttt{split()}.

Another interesting function is \ttt{reshape()}, which can turn a "long" table
into a "wide" one, by taking distinct values from a column and turning them
into separate columns. We better show this on another table, \ttt{x\$runattrs},
which contains the run attributes for each simulation run:

\begin{verbatim}
> x$runattrs
                                           runid       attrname                               attrvalue
1  PureAlohaExperiment-0-20100427-16:30:15-25274     configname                     PureAlohaExperiment
2  PureAlohaExperiment-0-20100427-16:30:15-25274       datetime                       20100427-16:30:15
3  PureAlohaExperiment-0-20100427-16:30:15-25274     experiment                     PureAlohaExperiment
4  PureAlohaExperiment-0-20100427-16:30:15-25274        inifile                             omnetpp.ini
5  PureAlohaExperiment-0-20100427-16:30:15-25274  iterationvars                   $numHosts=10, $mean=1
...
17 PureAlohaExperiment-1-20100427-16:30:16-25283     configname                     PureAlohaExperiment
18 PureAlohaExperiment-1-20100427-16:30:16-25283       datetime                       20100427-16:30:16
19 PureAlohaExperiment-1-20100427-16:30:16-25283     experiment                     PureAlohaExperiment
20 PureAlohaExperiment-1-20100427-16:30:16-25283        inifile                             omnetpp.ini
21 PureAlohaExperiment-1-20100427-16:30:16-25283  iterationvars                   $numHosts=10, $mean=1
...
\end{verbatim}

Reshaping can turn this into a table where each run has one row, and the table has
a "runid" column plus columns like "configname", "datatime", "experiment", etc.,
one for each run attribute. (The output has been broken to multiple lines, due to the 
large number of columns.)

\begin{verbatim}
> reshape(x$runattrs, idvar="runid", timevar="attrname", direction="wide")

                                           runid attrvalue.configname attrvalue.datetime attrvalue.experiment
1  PureAlohaExperiment-1-20100427-16:30:16-25283  PureAlohaExperiment  20100427-16:30:16  PureAlohaExperiment
17 PureAlohaExperiment-2-20100427-16:30:17-25289  PureAlohaExperiment  20100427-16:30:17  PureAlohaExperiment
33 PureAlohaExperiment-0-20100427-16:30:15-25274  PureAlohaExperiment  20100427-16:30:15  PureAlohaExperiment

   attrvalue.inifile attrvalue.iterationvars             attrvalue.iterationvars2 attrvalue.mean attrvalue.measurement
1        omnetpp.ini   $numHosts=10, $mean=1 $numHosts=10, $mean=1, $repetition=1              1 $numHosts=10, $mean=1
17       omnetpp.ini   $numHosts=10, $mean=2 $numHosts=10, $mean=2, $repetition=0              2 $numHosts=10, $mean=2
33       omnetpp.ini   $numHosts=10, $mean=1 $numHosts=10, $mean=1, $repetition=0              1 $numHosts=10, $mean=1

   attrvalue.network attrvalue.numHosts attrvalue.processid attrvalue.repetition attrvalue.replication
1              Aloha                 10               25283                    1                    #1
17             Aloha                 10               25289                    0                    #0
33             Aloha                 10               25274                    0                    #0

   attrvalue.resultdir attrvalue.runnumber attrvalue.seedset
1              results                   1                 1
17             results                   2                 2
33             results                   0                 0
\end{verbatim}

A similar command can be used to produce a table where each scalar is a column
and there are rows for distinct (run ID, module name) pairs, or a table where
each run has maps to one row and columns names are concatenations of a module name
and a scalar name.

The following lines implement the second one. The code first adds a new "qname"
column to a copy of the scalars table to contain "modulename/scalarname" strings
(\ttt{paste()} is R for string concatenation), and executes \ttt{reshape()} on it,
dropping the unnecessary columns:

\begin{verbatim}
> xs <- x$scalars
> xs$qname <- paste(xs$module, xs$name, sep="/")
> reshape(xs, idvar="runid", timevar="qname", direction="wide", drop=c("resultkey", "name", "module", "file"))
\end{verbatim}


\section{Combining Data Frames}

We have seen that the loaded dataset contains several tables (data frames):

\begin{verbatim}
> names(x)
[1] "runattrs"   "fileruns"   "scalars"    "vectors"    "statistics"
[6] "fields"     "bins"       "params"     "attrs"
\end{verbatim}

As for some queries it is necessary to combine the information in these tables,
let's examine what they contain. We have seen "scalars": it contains output scalars,
together with the file and run they were loaded from.

"vectors" is similar: it contains the list of output vectors recorded, together
with the file and run they were loaded from. However, it does not contain the
actual data: as the amount of data in vectors can be quite large, it can be read
with a separate command (\ttt{loadVectors()}) that also supports some filtering
and processing that can reduce the amount of data to be read into memory.

"statistics" contains the names of recorded statistical summaries and histograms,
together with file and run info. The actual data are in the "fields" table that
contains fields like "count", "sum", "sqrsum", "min", "max", etc. Histogram bins
are in the "bins" table.

Attributes (metadata annotation such as "unit", "interpolationmode", "title") for
scalars, vectors, and statistics/histograms are in the "attrs" table.

Result items (i.e. scalars, vectors, and statistics/histograms) are identified with
a unique "resultkey" column in their tables; "resultkey" is unique within the dataset.
The "fields", "bins" and "attrs" tables also contain a "resultkey" column to
identify which row belongs to which vector or scalar or statistic.

Files are identified with the file name, and runs with the run ID ("runid" column).
They are in many-to-many relationship: one simulation run may write to more than
one file (typically a scalar and a vector file, or several of each in the case of
distributed simulations), and a file may contain data from several runs (although it
usually does not). The "fileruns" table contains the valid (file, run) pairs,
i.e. which run has data in which files.

Finally, there is a "runattrs" table that contains the values of run attributes,
where the "runid" column identifies which run each row belongs to.

The tables have the following columns (we apply the \ttt{names()} function to
elements of the dataset to get this info):

\begin{verbatim}
> sapply(x,names)

$runattrs
[1] "runid"     "attrname"  "attrvalue"

$fileruns
[1] "runid" "file"

$scalars
[1] "resultkey" "runid"     "file"      "module"    "name"      "value"

$vectors
[1] "resultkey" "runid"     "file"      "vectorid"  "module"    "name"

$statistics
[1] "resultkey" "runid"     "file"      "module"    "name"

$fields
[1] "resultkey"  "fieldname"  "fieldvalue"

$bins
[1] "resultkey"  "lowerbound" "upperbound" "count"

$params
[1] "runid" "name"  "value"

$attrs
[1] "attrtype"  "resultkey" "attrname"  "attrvalue"
\end{verbatim}

You can see that there is no dedicated "runs" or "files" table; the list of
files and runs can be obtained from the "fileruns" table:

\begin{verbatim}
> levels(x$fileruns$file)
[1] "PureAlohaExperiment-0.sca" "PureAlohaExperiment-1.sca" "PureAlohaExperiment-2.sca"

> levels(x$fileruns$runid)
[1] "PureAlohaExperiment-0-20100427-16:30:15-25274" "PureAlohaExperiment-1-20100427-16:30:16-25283"
[3] "PureAlohaExperiment-2-20100427-16:30:17-25289"
\end{verbatim}

R has built-in support for "joining" tables (in the SQL sense). The corresponding
R function is called \ttt{merge()}, and one of its pleasant properties is that it
automatically figures out which columns to join by (it uses the column names that
occur in both tables).

For example, merging the "fileruns" and "runattrs" tables results in the following:

\begin{verbatim}
> merge(x$fileruns, x$runattrs)
                                           runid                      file       attrname              attrvalue
1  PureAlohaExperiment-0-20100427-16:30:15-25274 PureAlohaExperiment-0.sca      processid                  25274
2  PureAlohaExperiment-0-20100427-16:30:15-25274 PureAlohaExperiment-0.sca     repetition                      0
3  PureAlohaExperiment-0-20100427-16:30:15-25274 PureAlohaExperiment-0.sca    replication                     #0
4  PureAlohaExperiment-0-20100427-16:30:15-25274 PureAlohaExperiment-0.sca     configname    PureAlohaExperiment
5  PureAlohaExperiment-0-20100427-16:30:15-25274 PureAlohaExperiment-0.sca       datetime      20100427-16:30:15
6  PureAlohaExperiment-0-20100427-16:30:15-25274 PureAlohaExperiment-0.sca     experiment    PureAlohaExperiment
7  PureAlohaExperiment-0-20100427-16:30:15-25274 PureAlohaExperiment-0.sca        inifile            omnetpp.ini
8  PureAlohaExperiment-0-20100427-16:30:15-25274 PureAlohaExperiment-0.sca  iterationvars  $numHosts=10, $mean=1
...
17 PureAlohaExperiment-1-20100427-16:30:16-25283 PureAlohaExperiment-1.sca     configname    PureAlohaExperiment
18 PureAlohaExperiment-1-20100427-16:30:16-25283 PureAlohaExperiment-1.sca       datetime      20100427-16:30:16
19 PureAlohaExperiment-1-20100427-16:30:16-25283 PureAlohaExperiment-1.sca     experiment    PureAlohaExperiment
20 PureAlohaExperiment-1-20100427-16:30:16-25283 PureAlohaExperiment-1.sca        inifile            omnetpp.ini
21 PureAlohaExperiment-1-20100427-16:30:16-25283 PureAlohaExperiment-1.sca  iterationvars  $numHosts=10, $mean=1
...
\end{verbatim}

and merging "statistics" and "fields" results in this:

\begin{verbatim}
> merge(x$statistics, x$fields)
   resultkey                                         runid                      file       module                            name fieldname   fieldvalue
1         27 PureAlohaExperiment-0-20100427-16:30:15-25274 PureAlohaExperiment-0.sca Aloha.server       collisionLength:histogram     count 1.226600e+04
2         27 PureAlohaExperiment-0-20100427-16:30:15-25274 PureAlohaExperiment-0.sca Aloha.server       collisionLength:histogram       max 8.734217e-01
3         27 PureAlohaExperiment-0-20100427-16:30:15-25274 PureAlohaExperiment-0.sca Aloha.server       collisionLength:histogram      mean 1.969656e-01
4         27 PureAlohaExperiment-0-20100427-16:30:15-25274 PureAlohaExperiment-0.sca Aloha.server       collisionLength:histogram       min 9.916972e-02
5         27 PureAlohaExperiment-0-20100427-16:30:15-25274 PureAlohaExperiment-0.sca Aloha.server       collisionLength:histogram    sqrsum 5.692561e+02
6         27 PureAlohaExperiment-0-20100427-16:30:15-25274 PureAlohaExperiment-0.sca Aloha.server       collisionLength:histogram    stddev 8.726083e-02
7         27 PureAlohaExperiment-0-20100427-16:30:15-25274 PureAlohaExperiment-0.sca Aloha.server       collisionLength:histogram       sum 2.415980e+03
8         28 PureAlohaExperiment-0-20100427-16:30:15-25274 PureAlohaExperiment-0.sca Aloha.server collisionMultiplicity:histogram     count 1.226600e+04
9         28 PureAlohaExperiment-0-20100427-16:30:15-25274 PureAlohaExperiment-0.sca Aloha.server collisionMultiplicity:histogram       max 1.900000e+01
10        28 PureAlohaExperiment-0-20100427-16:30:15-25274 PureAlohaExperiment-0.sca Aloha.server collisionMultiplicity:histogram      mean 3.255177e+00
\end{verbatim}

Then you can filter the rows of the resulting table by indexing with logical indixes.
For example, to keep the "count" and "mean" colums only, you could enter:

\begin{verbatim}
> sf <- merge(x$statistics, x$fields)
> sf <- sf[sf$fieldname=="count" | sf$fieldname=="mean",]
\end{verbatim}

Note that we used "\ttt{|}" and not "\ttt{||}"! The latter operator also exists,
but is not useful here.

Now, if you would like to augment the "statistics" table with the mean, sum,
standard deviation, min, max, etc. columns, the following two lines will do it
for you (we use \ttt{reshape()} that's been covered before):

\begin{verbatim}
> widefields <- reshape(x$fields, idvar="resultkey", timevar="fieldname", direction="wide")
> merge (x$statistics, widefields)
  resultkey                                         runid                      file       module
1        27 PureAlohaExperiment-0-20100427-16:30:15-25274 PureAlohaExperiment-0.sca Aloha.server
2        28 PureAlohaExperiment-0-20100427-16:30:15-25274 PureAlohaExperiment-0.sca Aloha.server
3        29 PureAlohaExperiment-1-20100427-16:30:16-25283 PureAlohaExperiment-1.sca Aloha.server
4        30 PureAlohaExperiment-1-20100427-16:30:16-25283 PureAlohaExperiment-1.sca Aloha.server
5        31 PureAlohaExperiment-2-20100427-16:30:17-25289 PureAlohaExperiment-2.sca Aloha.server
6        32 PureAlohaExperiment-2-20100427-16:30:17-25289 PureAlohaExperiment-2.sca Aloha.server
                             name fieldvalue.count fieldvalue.max fieldvalue.mean fieldvalue.min fieldvalue.sqrsum
1       collisionLength:histogram            12266      0.8734217       0.1969656     0.09916972          569.2561
2 collisionMultiplicity:histogram            12266     19.0000000       3.2551769     2.00000000       163976.0000
3       collisionLength:histogram            12368      1.1932558       0.1980867     0.09919746          581.5508
4 collisionMultiplicity:histogram            12368     22.0000000       3.2783797     2.00000000       168555.0000
5       collisionLength:histogram             5875      0.6155766       0.1682003     0.09919544          183.2873
6 collisionMultiplicity:histogram             5875      9.0000000       2.4910638     2.00000000        40809.0000
  fieldvalue.stddev fieldvalue.sum
1        0.08726083      2415.9798
2        1.66504791     39928.0000
3        0.08822078      2449.9358
4        1.69728446     40547.0000
5        0.05391656       988.1766
6        0.86077865     14635.0000
\end{verbatim}

Augmenting the "scalars" table with, say, the "experiment", "measurement" and "replication"
run attributes can be done in a similar way.


\section{Using SQL}

If you are familiar with relational databases, you might prefer to use SQL instead of
R expressions. The "sqldf" R package lets you view data frames as database tables,
and query them with SELECT statements. An example session:

\begin{verbatim}
> install.packages("sqldf")
...
> require(sqldf)
Loading required package: sqldf
...
> attach(x)
> sqldf("select runid, module, name, value from scalars")
                                           runid       module                    name        value
1  PureAlohaExperiment-0-20100427-16:30:15-25274            .                    mean 1.000000e+00
2  PureAlohaExperiment-0-20100427-16:30:15-25274            .                numHosts 1.000000e+01
3  PureAlohaExperiment-0-20100427-16:30:15-25274 Aloha.server                duration 5.400007e+03
4  PureAlohaExperiment-0-20100427-16:30:15-25274 Aloha.server    collisionLength:mean 1.969656e-01
...

> sqldf("select s.runid, s.module, s.name, f.fieldname, f.fieldvalue from statistics s, fields f
where s.resultkey==f.resultkey and f.fieldname=='mean'")
                                          runid       module                            name fieldname fieldvalue
1 PureAlohaExperiment-0-20100427-16:30:15-25274 Aloha.server       collisionLength:histogram      mean  0.1969656
2 PureAlohaExperiment-0-20100427-16:30:15-25274 Aloha.server collisionMultiplicity:histogram      mean  3.2551769
3 PureAlohaExperiment-1-20100427-16:30:16-25283 Aloha.server       collisionLength:histogram      mean  0.1980867
4 PureAlohaExperiment-1-20100427-16:30:16-25283 Aloha.server collisionMultiplicity:histogram      mean  3.2783797
...
\end{verbatim}


\section{Plotting Bar Charts}

Bar charts can be plot with the \ttt{plotBarChart()} function of the \ttt{omnetpp}
package. This function expects two arguments. The first argument should be 
a numeric matrix with the data to be plot, where rows are the groups of the 
bar chart and columns are the bars. The second argument is a properties list
which defines the title of the chart, axis labels, controls the legend, etc.
The property names and values are intended to be as close to the properties 
used by the IDE's Analysis Editor as possible.

The matrix containing the chart data can be created in various ways, using R's 
data manipulation facilities. The package also contains a \ttt{makeBarChartDataset()}
function that extracts the data from scalars in a dataset. An example:

\begin{verbatim}
> bcd <- makeBarChartDataset(x, rows=c('module'), columns='name')
> plotBarChart(bcd, list(Legend.Display="true"))
\end{verbatim}

TODO: Refine this section of the tutorial.

TODO: The example in the help page of \ttt{makeBarChartDataset()} does not work.


\section{Plotting Histograms}

The \ttt{loadDataset()} function loads histograms into the "statistics" and "bins" tables,
which is not really ready for immediate consumption. The \ttt{makeHistograms()} function
can be used to convert histograms in a dataset into a list of R histogram structures:

\begin{verbatim}
> hh <- makeHistograms(x)
\end{verbatim}

The function tries to give unique names to the histograms, assembled from various columns
of the "statistics" table:

\begin{verbatim}
> names(hh)
> names(makeHistograms(x))
[1] "PureAlohaExperiment-0-20100427-16:30:15-25274 collisionLength:histogram"
[2] "PureAlohaExperiment-0-20100427-16:30:15-25274 collisionMultiplicity:histogram"
[3] "PureAlohaExperiment-1-20100427-16:30:16-25283 collisionLength:histogram"
...
\end{verbatim}

It is also possible to specify with a format string how the names should be created:

\begin{verbatim}
> hh <- makeHistograms(x, "${module} ${name}")
> names(hh)
[1] "Aloha.server collisionLength:histogram"      
[2] "Aloha.server collisionMultiplicity:histogram"
[3] "Aloha.server collisionLength:histogram"      
...
\end{verbatim}

TODO: The format string currently does not recognize \ttt{\$\{experiment\}}, 
\ttt{\$\{measurement\}}, \ttt{\$\{replication\}}, \ttt{\$\{configname\}},
\ttt{\$\{runnumber\}}, \ttt{\$\{network\}} and other run attributes.

The resulting histogram objects can be plot using R's "plot" function. The
following command plots the first histogram:

\begin{verbatim}
> plot(hh[[1]])
\end{verbatim}

If you want to review all histograms (click through the plots), you can do it with
the following command:

\begin{verbatim}
> par(ask=TRUE)
> sapply(hh, plot)
\end{verbatim}

The first line is only needed once per session, it tells R to wait between the plots
(otherwise the histograms would just whizz by).


\section{Plotting Vectors}

Let's begin with loading a vector file:

\begin{verbatim}
> x <-loadDataset("OneFifo*.vec")
> x$vectors
   resultkey                             runid          file vectorid            module                          name
1          0 OneFifo-0-20081210-11:48:12-11130 OneFifo-0.vec        9 SimpleQueue.queue                 queueing time
2          1 OneFifo-0-20081210-11:48:12-11130 OneFifo-0.vec        8 SimpleQueue.queue                        length
3          2 OneFifo-0-20081210-11:48:12-11130 OneFifo-0.vec        0  SimpleQueue.sink                total lifetime
4          3 OneFifo-0-20081210-11:48:12-11130 OneFifo-0.vec        1  SimpleQueue.sink           total queueing time
5          4 OneFifo-0-20081210-11:48:12-11130 OneFifo-0.vec        2  SimpleQueue.sink number of queue nodes visited
6          5 OneFifo-0-20081210-11:48:12-11130 OneFifo-0.vec        3  SimpleQueue.sink            total service time
7          6 OneFifo-0-20081210-11:48:12-11130 OneFifo-0.vec        4  SimpleQueue.sink                   total delay
8          7 OneFifo-0-20081210-11:48:12-11130 OneFifo-0.vec        5  SimpleQueue.sink number of delay nodes visited
9          8 OneFifo-0-20081210-11:48:12-11130 OneFifo-0.vec        6  SimpleQueue.sink                    generation
10         9 OneFifo-0-20081210-11:48:12-11130 OneFifo-0.vec        7 SimpleQueue.queue                  dropped jobs
\end{verbatim}

\ttt{loadDataset()} only loads the vector declarations. The actual vector data can be
loaded wit the \ttt{loadVectors()} function:

\begin{verbatim}
> v <- loadVectors(x, NULL)
\end{verbatim}

If you get the error \ttt{indexed vector file reader: index file is not up to date}, then
you can use the \ttt{scavetool} program, part of OMNeT++, to index the file:

\begin{verbatim}
$ scavetool index OneFifo-0.vec
\end{verbatim}

The \ttt{loadVectors()} function creates a list with three data frames:

\begin{verbatim}
> names(v)
[1] "vectors"    "vectordata" "attributes"
> v
$vectors
   resultkey          file vectorid            module                          name
1          0 OneFifo-0.vec        9 SimpleQueue.queue                 queueing time
2          1 OneFifo-0.vec        8 SimpleQueue.queue                        length
3          2 OneFifo-0.vec        0  SimpleQueue.sink                total lifetime
...

$vectordata
   resultkey eventno         x          y
1          0       2  1.199993  0.0000000
2          0       5  2.540825  0.2951858
3          0      15  4.565747  1.9815322
4          0      19  6.454965  3.5643205
5          0      25 10.279441  6.1744295
...
10         1       2  1.199993  0.0000000
11         1       4  2.245639  1.0000000
12         1       5  2.540825  0.0000000
13         1       8  2.584215  1.0000000
14         1      10  2.890644  2.0000000
15         1      12  4.105012  3.0000000
16         1      14  4.436527  4.0000000
...
30         2       6  2.540825  1.3408317
31         2      16  4.565747  2.3201081
32         2      20  6.454965  3.8707496
33         2      26 10.279441  7.3887973
...
$attrs
[1] resultkey attrname  attrvalue
\end{verbatim}

The \ttt{vectordata} component contains data from all vectors.
One can use \ttt{split()} to split it to separate vectors:

\begin{verbatim}
> split(v$vectordata, v$vectordata$resultkey)
$`0`
   resultkey eventno         x          y
1          0       2  1.199993  0.0000000
2          0       5  2.540825  0.2951858
3          0      15  4.565747  1.9815322
4          0      19  6.454965  3.5643205
...

$`1`
    resultkey eventno         x y
10          1       2  1.199993 0
11          1       4  2.245639 1
12          1       5  2.540825 0
13          1       8  2.584215 1
...
\end{verbatim}

This can be turned into a line chart with the \ttt{plotLineChart()} function:

\begin{verbatim}
> vs <- split(v$vectordata, v$vectordata$resultkey)
> plotLineChart(vs, list())
\end{verbatim}

The second argument to \ttt{plotLineChart()} is a properties list, it can be used
to set things like axis titles, chart title, grid lines, etc. The property names and 
values are intended to be as close to the properties used by the IDE's Analysis Editor 
as possible, to facilitate future interoperability. An example:

\begin{verbatim}
> plotLineChart(vs, list(X.Axis.Title='t', Y.Axis.Title='value'))
\end{verbatim}

Otherwise \ttt{plotLineChart()} just calls R's \ttt{plot.new()}, \ttt{plot.window()}, 
\ttt{lines()}, \ttt{title()} and \ttt{legend()} functions with the appropriate parameters
(check it with \ttt{fix(plotLineChart)}!

TODO: Currently the legend is not really good: as line titles come from the names of
the split vector data, it simply says "0", "1", "2", etc. Those names can be set 
by assigning \ttt{names(vs)} (e.g. after issuing \ttt{names(vs) <- paste("Line", names(vs))}
the legend will say "Line 0", "Line 1", etc), and preferably there should be a function
that splits "vectordata" and also assigns good names to individual vectors.

The \ttt{loadVectors()} function also supports filtering. For example, the following
command applies a \ttt{winavg(10)} filter to the vectors:

\begin{verbatim}
> loadVectors(x, NULL, apply(winavg(10)))
\end{verbatim}

To see the list of available filters, type \ttt{?filters}.

The filtering occurs while the data are read from disk, and only the result is 
stored in memory. This allows large vector files to be processed, where even individual 
vectors contain huge amounts of data.


\section{Summary}

This concludes our short tutorial. Please help us make this tutorial and the R package useful,
and contribute suggestions and improvements!



